


\chapter{Nombre del capítulo}\markboth{CAPÍTULO 4: Nombre del capítulo...}{\small CAPÍTULO 4: Nombre del capítulo...}
\label{EtiquetaCap4}

%\def\baselinestretch{2.0}
% \begin{quote}
 En este capítulo...
 %\end{quote}
\section{Cómo citar}\label{Cap2}
\begin{itemize}
  \item \textbf{Citas cortas} (hasta 40 palabras):Se incluyen en el texto con comillas.
  \item \textbf{Citas largas} (más de 40 palabras): Se escriben en párrafo separado sin comillas.
\end{itemize}
\subsection{Citas textuales}

\begin{itemize}
  \item \textbf{Narrativa}.
  
  \underline{Cita corta}.
   
   Como menciona \cite{apostol1976analisis}, ``\emph{la idea de expresar geométricamente los números complejos como puntos de un plano fue formulada por Gauss en su disertación de 1799 e, independiente, por Argand en 1806.}'' (p. 21). Sin embargo ...
  
  \underline{Cita larga}
  
  Como menciona \cite{apostol1976analisis}:
  \begin{itemize}
    \item [] La idea de expresar geométricamente los números complejos como puntos de un plano fue formulada por Gauss en su disertación de 1799 e, independiente, por Argand en 1806. Más tarde Gauss ideó la expresión un tanto desafortunada de ''número complejo''. Los números complejos admiten otras representaciones geométricas. En vez de utilizar puntos de un plano, se pueden utilizar puntos de otras superficies. (p. 22)
  \end{itemize}

  

  \item \textbf{Con paréntesis}.
  
  Los números complejos, pueden representarse como puntos de un plano, ``\emph{la idea de expresar geométricamente los números complejos como puntos de un plano fue formulada por Gauss en su disertación de 1799 e, independiente, por Argand en 1806.}'' \cite[p. 21]{apostol1976analisis}, 
  
\end{itemize}

\subsection{Citas parafraseadas}
\begin{itemize}
  \item \textbf{Narrativa}.
  
  De acuerdo a \cite{apostol1976analisis}, otra representación geométrica de los números complejos es la llamada \textbf{proyección estereográfica} que consiste en proyectar los puntos del polo norte de la esfera sobre el plano tangente en el polo de dicha esfera. (p. 22)
  \item \textbf{Con paréntesis}.
  
  Otra representación geométrica de los números complejos es la llamada \textbf{proyección estereográfica} que consiste en proyectar los puntos del polo norte de la esfera sobre el plano tangente en el polo de dicha esfera.\cite[p.22]{apostol1976analisis}
\end{itemize}
\section{Tipos de referencias}
\subsection{Revistas}
Smith, T., and Jones, M.: ‘Titulo del articulo’, Nombre de la Revista, Volumen (Número),  pp. inicial–final (Año)
\subsection{Congresos o reuniones}
Jones, L., and Brown, D.: ‘Titulo del articulo’. Nombre del volumen del congreso, Paginas, Ciudad, País, Fechas de la reunion, Año.
\subsection{Norma}
Institución responsable, Título de la norma, Designación fija para esta norma seguido de año de adopción original o en el caso de revisión, del año de la última revisión, Año de publicación
\subsection{Páginas o documentos electrónicos en la red}
http://www.theiet.org, Accedido en Abril 2006
\subsection{Patentes}
\begin{itemize}
  \item Brown, F.: ‘Titulo de la patente’. País de la patente, Número de patente, Fecha de la patente
  \item Smith, D., and Hodges, J.: British Patent Application 98765, 1925
\end{itemize}
\subsection{Tesis doctorales}
 Abbott, N.L.: ‘Título de la tesis’. Tesis Doctoral, Universidad, Año

%\begin{tabular}{|c|c|}
%  \hline
%  % after \\: \hline or \cline{col1-col2} \cline{col3-col4} ...
%  a & b \\
%  c & d \\
%  \hline\label{TablaEnCap4}
%\end{tabular}
