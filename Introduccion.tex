\chapter*{Introducción}
\markboth{INTRODUCCIÓN}{\small INTRODUCCIÓN}
\addcontentsline{toc}{chapter}{Introducción}
\setcounter{page}{1}


\noindent La evaluación educativa es un tema importante a nivel mundial. Los gobiernos \corregido{de distintos países,} preocupados por este tema, recurren a diversos organismos nacionales e internacionales como la Asociación Internacional para la Evaluación del Rendimiento Educativo (IEA), la Organización para la Cooperación y el Desarrollo Económico (OCDE) o la Comisión Nacional para la Mejora Continua de la Educación (MEJOREDU), \corregido{los cuales aplican} pruebas como TIMSS, PISA o PLANEA, \corregido{que} tienen como objetivo conocer el estado \corregido{actual de} la educación en cada país.

\noindent En México, desde 2006 \corregido{con} la prueba ENLACE, \corregido{existía} un mecanismo que permitía conocer información diagnóstica del nivel de logro académico \corregido{en} las asignaturas de español y matemáticas, \corregido{que fue sustituida} por PLANEA en 2015.

\noindent Los resultados en México \corregido{en estas pruebas} han sido bajos, \corregido{con la mayoría ubicados} en el nivel I. Esto representa una preocupación para las instituciones educativas, \corregido{que} consideran \corregido{estos resultados como} un reto \corregido{y un} desafío para la educación media superior. A pesar de la importancia de estas evaluaciones, desde 2022 \corregido{no se han aplicado} pruebas estandarizadas como PLANEA o PISA; \corregido{sin embargo, este es un tema que no debe perderse de vista.}

\noindent En el presente trabajo de tesis, se aborda esta problemática \corregido{buscando} una estrategia para mejorar el desempeño de los estudiantes, mediante el diseño e implementación de una propuesta didáctica \corregido{que mejore su desempeño} en la resolución de reactivos tipo PLANEA en Educación Media Superior. En particular, se trabajó con estudiantes del plantel 06 de Contla, en el campo disciplinar de cálculo integral \corregido{de} sexto semestre.

\noindent \corregido{A modo de guía, a continuación se describe la estructura del trabajo de tesis:}

\noindent En el \textbf{Capítulo 1} se presenta el planteamiento del problema, las preguntas de investigación, los objetivos, la hipótesis y la justificación.

\noindent El \textbf{Capítulo 2} expone la parte metodológica, \corregido{incluyendo} conceptos como: evaluación, tipos de evaluación, examen, prueba, tipos de reactivos, aprendizaje esperado, \corregido{además de} información relevante sobre la prueba PLANEA, sus propósitos, consideraciones metodológicas, niveles de logro, \corregido{su} relación con el marco curricular común \corregido{y aspectos de} diseño instruccional.

\noindent El \textbf{Capítulo 3} describe la aplicación del método ADDIE (Análisis, Diseño, Desarrollo, Implementación y Evaluación) \corregido{detallando} cada fase. \corregido{Se analizan} las pruebas PLANEA 2016, 2017 y 2020, \corregido{construyéndose} tablas que clasifican los reactivos según su eje temático, el tema correspondiente en el programa de estudios del campo disciplinar de matemáticas de la Dirección General de Bachillerato (DGB) y el número de reactivos por tema. \corregido{La última sección especifica el} lugar donde se realizó la investigación.

\noindent El \textbf{Capítulo 4} presenta los resultados de la implementación \corregido{del material}, \corregido{los resultados} de la evaluación aplicada a los alumnos \corregido{y su} análisis para determinar el nivel de logro.

\noindent Finalmente, el \textbf{Capítulo 5} establece la validación de la hipótesis, las conclusiones, hallazgos \corregido{y} recomendaciones.