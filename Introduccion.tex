\chapter*{Introducción}
\markboth{INTRODUCCIÓN}{\small INTRODUCCIÓN}
\addcontentsline{toc}{chapter}{Introducción}
\setcounter{page}{1}


\noindent El análisis de bioseñales constituye un campo de investigación interdisciplinario en el que convergen la matemática aplicada, la fisiología y la ingeniería biomédica, con el propósito de interpretar cuantitativamente los fenómenos eléctricos asociados a la actividad neuromuscular. Dichas señales, capturadas mediante instrumentos de registro como la electromiografía (EMG), permiten acceder a información de alto valor sobre los mecanismos que gobiernan la excitabilidad de los tejidos y la coordinación funcional entre el sistema nervioso y los efectores musculares. En este sentido, el estudio de las bioseñales no solo proporciona herramientas diagnósticas y experimentales, sino que también abre la posibilidad de modelar procesos fisiológicos complejos bajo marcos matemáticos rigurosos y reproducibles. 

\noindent Dentro de este contexto, la uretra —estructura esencial en la fisiología miccional— ha sido objeto de estudios neurofisiológicos orientados a comprender la dinámica de su control motor y reflejo. En particular, la actividad electromiográfica del esfínter externo de la uretra (EEU) representa un indicador directo de la función neuromuscular responsable de mantener la continencia urinaria. Investigaciones experimentales realizadas en modelos animales, especialmente en conejas, han permitido establecer la relación entre las raíces nerviosas lumbosacras y el control del EEU. En tales modelos, la avulsión de la raíz ventral lumbosacra (L6–S1) provoca una alteración significativa en los patrones electromiográficos, reflejando la pérdida de la inervación motora del esfínter y, por ende, la disfunción miccional subsecuente (véase [Corona-Quintanilla et al., año]). Dicho modelo constituye una referencia experimental fundamental para la comprensión de los mecanismos neurales que regulan la presión uretral y la coordinación del sistema urinario inferior.

\noindent El Centro Tlaxcala de Biología de la Conducta (CTBC) ha desarrollado líneas de investigación orientadas al análisis de la actividad electromiográfica uretral en conejas bajo condiciones de estimulación eléctrica controlada. Este enfoque ha permitido obtener registros detallados de la presión uretral (PU) en tres regiones anatómicas —proximal, medial y distal—, cada una sometida a trenes de estímulos eléctricos triplicados. Los datos experimentales se estructuran en ventanas de análisis de quince segundos (Basal, A-inicial, A-mayor y A-final), las cuales permiten capturar la evolución temporal de la respuesta fisiológica ante la estimulación. Estos registros constituyen la materia prima para la aplicación de métodos matemáticos y estadísticos avanzados destinados a la extracción e interpretación de los patrones de respuesta neuromuscular.

\noindent Desde el punto de vista analítico, la Transformada Rápida de Fourier (FFT, por sus siglas en inglés) representa una de las herramientas más potentes para el estudio espectral de bioseñales. Esta técnica permite descomponer la señal temporal en sus componentes frecuenciales, revelando las estructuras oscilatorias subyacentes y proporcionando una descripción más completa del fenómeno fisiológico observado. En la presente investigación, la FFT se aplica bajo condiciones parametrizadas mediante una ventana de Hamming, unidades de visualización en Vpk/rHz y un ancho de bin de 15, lo que posibilita reducir el leakage espectral y mejorar la resolución frecuencial. A partir de los valores espectrales obtenidos, se propone un análisis estadístico orientado a identificar diferencias significativas entre las regiones anatómicas y las ventanas temporales de estimulación.

\noindent El desarrollo de esta metodología encuentra su principal soporte en la integración de herramientas matemáticas, computacionales y estadísticas dentro de un marco reproducible de análisis. En este sentido, el lenguaje de programación Python se consolida como un recurso idóneo por su naturaleza abierta, su vasta comunidad científica y la disponibilidad de bibliotecas especializadas (NumPy, SciPy, Matplotlib, Pandas, entre otras) que facilitan tanto el procesamiento numérico como la visualización de datos experimentales. El diseño de un marco computacional en Python no solo busca optimizar el análisis de bioseñales, sino también fomentar la transparencia y la replicabilidad científica, principios fundamentales en la investigación contemporánea.

\noindent En suma, la presente tesis se inscribe en la intersección entre la matemática aplicada y la neurofisiología experimental, proponiendo el desarrollo de un software adaptable capaz de procesar, visualizar y analizar bioseñales electromiográficas provenientes de estudios experimentales en modelos animales. Este esfuerzo pretende no solo contribuir a la caracterización cuantitativa de la actividad del esfínter externo de la uretra, sino también ofrecer un entorno computacional que sirva como base para futuras investigaciones en el ámbito del análisis de señales biológicas, fortaleciendo el puente entre la modelación matemática y la comprensión de los procesos fisiológicos complejos.

%\noindent El \textbf{Capítulo 2} expone la parte metodológica, \corregido{incluyendo} conceptos como: evaluación, tipos de evaluación, examen, prueba, tipos de reactivos, aprendizaje esperado, \corregido{además de} información relevante sobre la prueba PLANEA, sus propósitos, consideraciones metodológicas, niveles de logro, \corregido{su} relación con el marco curricular común \corregido{y aspectos de} diseño instruccional.

%\noindent El \textbf{Capítulo 3} describe la aplicación del método ADDIE (Análisis, Diseño, Desarrollo, Implementación y Evaluación) \corregido{detallando} cada fase. \corregido{Se analizan} las pruebas PLANEA 2016, 2017 y 2020, \corregido{construyéndose} tablas que clasifican los reactivos según su eje temático, el tema correspondiente en el programa de estudios del campo disciplinar de matemáticas de la Dirección General de Bachillerato (DGB) y el número de reactivos por tema. \corregido{La última sección especifica el} lugar donde se realizó la investigación.

%\noindent El \textbf{Capítulo 4} presenta los resultados de la implementación \corregido{del material}, \corregido{los resultados} de la evaluación aplicada a los alumnos \corregido{y su} análisis para determinar el nivel de logro.

%\noindent Finalmente, el \textbf{Capítulo 5} establece la validación de la hipótesis, las conclusiones, hallazgos \corregido{y} recomendaciones.