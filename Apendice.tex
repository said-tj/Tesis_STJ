\chapter*{Apéndice (Opcional)}
\markboth{APÉNDICE}{\small APÉNDICE}
\addcontentsline{toc}{chapter}{Apéndice (Opcional)}
\label{Apendice}

\noindent \textbf{$($M. Asadi, etc. \cite{ASSOVARH2009}$)$}
Sea $E$ un espacio  de Banach real. Un subconjunto $\mathcal{P}\subset E$, no vacío, cerrado y convexo es llamado un \emph{cono} en $E$ si satisface lo siguiente:
\begin{itemize}
	\item [$i)$] $\mathcal{P}$ es cerrado, no vacío y $\mathcal{P}\neq \{0\}$,
	\item [$ii)$] $a,b\in \mathbb{R}$, $a,b\geq 0$ y $x,y\in \mathcal{P}$ implica que $a x+ b y \in \mathcal{P}$,
	\item [$iii)$] $x\in \mathcal{P}$ y $-x\in \mathcal{P}$ implica que $x=0$.
\end{itemize}

