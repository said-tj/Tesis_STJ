\chapter*{Conclusiones}
\markboth{CONCLUSIONES Y TRABAJOS FUTUROS}{\small CONCLUSIONES Y TRABAJOS FUTUROS}
\addcontentsline{toc}{chapter}{Conclusiones y trabajos futuros}

\noindent \textcolor[rgb]{0,0,1}{
	Este apartado trata de manera puntual la condición de aceptación o rechazo de la hipótesis. También se deberá referir los hechos más relevantes durante la investigación y los hallazgos o aportes principales, apoyados por la argumentación sobre el valor del estudio.}

\noindent \textcolor[rgb]{0,0,1}{
	A continuación presentamos ejemplos de cómo iniciar la redacción de las conclusiones:
	\begin{itemize}
		\item Con base en los resultados obtenidos en la presente investigación, se llegó a las siguientes conclusiones [\ldots]
		\item En cuanto a la teoría que respalda los resultados es importante enfatizar que [\ldots]
		\item La hipótesis quedó plenamente demostrada, pues los hallazgos permitieron confirmar [\ldots]
		\item El marco metodológico adoptado indica que [\ldots].
	\end{itemize}
}



\noindent \textbf{Trabajos Futuros}

\noindent \textcolor[rgb]{0,0,1}{
	Proponer trabajos futuros que se pueden realizar. Numerado, viñetas o texto.}

\noindent \textcolor[rgb]{0,0,1}{
	Ejemplo:\newline
	Como trabajos futuros que se pueden proponer y realizar, son los siguientes: 
	\begin{enumerate}
		\item Plantear el problema de distribución como uno en el área de Teoría de Redes.
		\item Considerar escalón más al modelo, que representaría la distribución desde las zonas de extracción a las refinerías.
		\item Subir a alguna plataforma especializada la base de datos creada en esta investigación.
		\item Proponer un modelo multi objetivo en la distribución de derivados del petróleo.
	\end{enumerate}
}


