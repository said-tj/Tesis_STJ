\chapter*{Resultados}
\markboth{ANÁLISIS E INTERPRETACIÓN DE RESULTADOS}{\small ANÁLISIS E INTERPRETACIÓN DE RESULTADOS}
\label{Resultados}
\addcontentsline{toc}{chapter}{Análisis e interpretación de resultados}

\noindent \textcolor[rgb]{0,0,1}{
	En este capítulo debes construir interpretaciones a partir de los datos obtenidos en la etapa anterior; para esto tendrás que explicar la validez del modelo metodológico aplicado y demostrar la objetividad de los instrumentos de medición utilizados.}

\noindent \textcolor[rgb]{0,0,1}{
	Te recomendamos que primero reportes de manera sintética los resultados y principales hallazgos de tu investigación, y después los describas de manera minuciosa, pero resérvate las conclusiones y recomendaciones para sus respectivos apartados. Relacionar siguiendo los objetivos específicos Se presentan los resultados logrados en forma cuantitativa. Para los objetivos que estén en vías de logro, o que no se lograron, se presentarán las causas correspondientes que expliquen la situación de cada uno según sea el caso.}

\noindent \textcolor[rgb]{0,0,1}{
	Esta sección se termina con una presentación de “Antes y Después”, mostrando los aspectos cuantitativos de cada problema resuelto. Esta parte puede incluir: tablas, gráficas, fotos, videos, etc.}\newpage

\noindent \textcolor[rgb]{0,0,1}{
	En la segunda, se reporta para que demuestres, que el proyecto fue rentable o benéfico para la empresa. Para tal efecto, se podrán utilizar demostraciones de: costo-­‐beneficio, diagramas de flujo de efectivo, rentabilidad por medio de la tasa de retorno de inversión. En esta parte deberán tomarse en cuenta todas las inversiones efectuadas, tales como: las adquisiciones, percepciones incluyendo las del pasante, horas estándar invertidas, pruebas, etc.}
