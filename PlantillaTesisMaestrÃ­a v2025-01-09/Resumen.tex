\chapter*{Resumen}
\markboth{RESUMEN}{\small RESUMEN}
\addcontentsline{toc}{chapter}{Resumen}

\setcounter{page}{1}
\noindent El presente trabajo muestra una metodología para preparar a los alumnos en la aplicación de pruebas estandarizadas, \corregido{en particular para mejorar} los resultados de la prueba PLANEA. \corregido{Se incluye} información teórico-conceptual sobre evaluación y su importancia para medir el desempeño académico \corregido{de los sistemas educativos} en diferentes países, en particular en México, a través de la prueba PLANEA.

\noindent Se plantea \corregido{un método} de mejora de los resultados mediante la implementación de \corregido{un banco de ejercicios} donde el alumno encuentre \corregido{un recurso} de trabajo para prepararse \corregido{previamente} antes de ser aplicada la prueba. \corregido{Se empleó} la metodología ADDIE, \corregido{mediante} procesos que incluyen \corregido{las fases de} análisis, diseño, desarrollo, implementación y evaluación.

\noindent Se recabó información de las pruebas PLANEA 2016 y 2017 para \corregido{organizar} los reactivos y \corregido{clasificarlos} en ejes, niveles y asignaturas \corregido{según los estándares de} PLANEA MS. Con esta información se construyó \corregido{un banco} de ejercicios tipo PLANEA y se propuso \corregido{un protocolo} de aplicación.

\noindent Para verificar \corregido{la efectividad} del \corregido{banco de ejercicios}, se llevó a cabo \corregido{una evaluación comparativa} diagnóstica y final. Los resultados se interpretaron a través de tablas y gráficas elaboradas por niveles de desempeño, \corregido{y se aplicó} la prueba de hipótesis $t$ de Student para \corregido{comparar las medias por niveles y determinar} si hubo una mejora \corregido{significativa} en los resultados.