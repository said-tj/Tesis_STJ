\chapter{Objeto de la investigación}
\markboth{CAPÍTULO 1. OBJETO DE LA INVESTIGACIÓN}{\small CAPÍTULO 1. OBJETO DE LA INVESTIGACIÓN}
\label{ObjetoInv}
 
\parskip=12pt
\section{Planteamiento del problema}\label{Seccion11}
\noindent El estudio cuantitativo de las bioseñales constituye una de las áreas más relevantes de la investigación biomédica contemporánea, pues permite traducir los fenómenos fisiológicos en estructuras matemáticas susceptibles de análisis, modelado y simulación. Dentro de este ámbito, las señales electromiográficas (EMG) representan un medio preciso para caracterizar la actividad neuromuscular en distintos órganos y tejidos. En particular, la actividad electromiográfica del esfínter externo de la uretra (EEU) se ha consolidado como un parámetro experimental esencial para evaluar los mecanismos de control neuromuscular asociados a la continencia y micción urinaria. Estudios recientes han evidenciado que lesiones en las raíces nerviosas lumbosacras —en especial la avulsión de la raíz ventral L6–S1— provocan una disminución drástica en la actividad electromiográfica del EEU, indicando una interrupción funcional en la transmisión motora hacia esta estructura (véase [Corona-Quintanilla et al., año]).

\noindent Este hallazgo ha posicionado el modelo de avulsión lumbosacra en conejas como un referente experimental para estudiar las bases neurofisiológicas del control miccional, permitiendo registrar de forma controlada la respuesta eléctrica uretral bajo distintos patrones de estimulación. No obstante, la complejidad intrínseca de los registros electromiográficos —caracterizados por su variabilidad, ruido fisiológico y dependencia del tiempo— plantea importantes desafíos analíticos. La interpretación de dichas señales requiere de metodologías matemáticas y computacionales capaces de extraer información significativa y reproducible, respetando la estructura espectral y temporal de los datos.

En el Centro Tlaxcala de Biología de la Conducta (CTBC) se han desarrollado protocolos experimentales que generan volúmenes considerables de datos de presión uretral (PU) y de actividad electromiográfica. Estos registros incluyen trenes de estimulación eléctrica aplicados a distintas regiones anatómicas de la uretra (proximal, medial y distal), segmentados en ventanas temporales que representan distintas fases de respuesta: Basal, A-inicial, A-mayor y A-final. Tal esquema proporciona un marco experimental robusto, pero al mismo tiempo produce una gran cantidad de información que debe ser tratada mediante herramientas especializadas de análisis de señales.

\parskip=15pt
\noindent En la actualidad, una parte significativa del procesamiento de bioseñales en el ámbito biomédico continúa realizándose mediante software propietario, el cual impone restricciones en cuanto a transparencia metodológica, personalización de parámetros y reproducibilidad científica. Esta dependencia limita la posibilidad de adaptar los procedimientos de análisis a las condiciones específicas de cada experimento, obstaculizando la incorporación de técnicas matemáticas avanzadas o ajustes precisos en la parametrización espectral —como la selección de funciones de ventana, resolución de bins o normalización de amplitudes—.

\noindent Aunado a ello, el tratamiento computacional de las señales electromiográficas exige un control riguroso sobre aspectos tales como la transformación al dominio frecuencial, la reducción de artefactos, y la comparación estadística entre regiones o condiciones experimentales. Sin un marco metodológico que integre de manera coherente las fases de adquisición, procesamiento y visualización, se corre el riesgo de obtener resultados parciales, difíciles de replicar y de escasa generalización científica.

\noindent En este escenario, se identifica una brecha metodológica y tecnológica: la ausencia de un entorno computacional abierto, reproducible y adaptable, que permita el análisis integral de bioseñales como las registradas en estudios de actividad uretral. Tal carencia no solo afecta la eficiencia del procesamiento y la consistencia de los resultados, sino que también limita el potencial de colaboración interdisciplinaria entre la matemática aplicada, la ingeniería biomédica y las neurociencias experimentales.

\noindent Por lo tanto, se plantea la necesidad de desarrollar un marco computacional basado en el lenguaje de programación Python, que integre algoritmos de procesamiento digital de señales, transformadas espectrales —como la Transformada Rápida de Fourier (FFT)—, herramientas de visualización y módulos de análisis estadístico. Este marco debe ser capaz de operar tanto en entornos de línea de comandos (CLI), para el manejo automatizado de grandes volúmenes de datos, como en entornos interactivos (Jupyter Notebooks), que favorezcan la exploración y documentación científica.

\noindent La formulación de este sistema computacional permitirá superar las limitaciones del software privativo, garantizar la reproducibilidad de los resultados y ofrecer un entorno versátil que pueda extenderse a otros tipos de bioseñales (como EEG o ECG). De este modo, se busca fortalecer el vínculo entre la investigación neurofisiológica experimental y la matemática aplicada, generando un modelo de análisis integral que contribuya al entendimiento profundo de los mecanismos fisiológicos subyacentes a la actividad uretral y a la consolidación de prácticas científicas transparentes, eficientes y sustentadas en fundamentos cuantitativos rigurosos.

\section{Justificación}

\noindent El estudio de las bioseñales electromiográficas constituye un eje estratégico para la comprensión de los mecanismos fisiológicos que sustentan la comunicación entre el sistema nervioso y los efectores musculares. En el caso particular del esfínter externo de la uretra (EEU), el análisis de su actividad eléctrica permite inferir el grado de integridad funcional de las vías nerviosas responsables del control miccional. Diversas investigaciones han demostrado que la avulsión de la raíz ventral lumbosacra (L6–S1) produce alteraciones significativas en la actividad electromiográfica del EEU, evidenciando el papel crítico de dichas raíces en la modulación motora del sistema urinario inferior (véase [Corona-Quintanilla et al., año]). Este tipo de modelos experimentales, desarrollados en el Centro Tlaxcala de Biología de la Conducta (CTBC), constituyen una base invaluable para el estudio de las interacciones neurofisiológicas que regulan la continencia y la función urinaria.

Sin embargo, la riqueza informativa contenida en los registros electromiográficos requiere de herramientas analíticas capaces de traducir los patrones eléctricos en estructuras matemáticas cuantificables. Las bioseñales, al ser inherentemente complejas, presentan componentes de ruido, variabilidad temporal y dinámica no lineal, lo que dificulta su interpretación mediante técnicas convencionales. Por tanto, se hace indispensable el diseño de un marco metodológico que permita abordar su análisis desde una perspectiva matemático-computacional, incorporando transformaciones, visualizaciones y métodos estadísticos que aporten objetividad y reproducibilidad a los resultados obtenidos.

\noindent El uso de software propietario en el análisis de bioseñales, aunque extendido en entornos biomédicos, impone limitaciones sustanciales a la investigación científica. La falta de transparencia en los algoritmos empleados, la imposibilidad de modificar parámetros internos y la ausencia de trazabilidad metodológica obstaculizan la replicabilidad de los experimentos y restringen la exploración de nuevos enfoques analíticos. En contraste, los entornos de código abierto —particularmente el ecosistema científico de Python— ofrecen una alternativa sólida, sustentada en la flexibilidad, la interoperabilidad y la validación colectiva. Bibliotecas como NumPy, SciPy, Matplotlib, Pandas y MNE-Python han demostrado su eficacia en el procesamiento de señales, el análisis espectral y la representación visual de datos fisiológicos, garantizando la precisión numérica y la transparencia metodológica necesarias para investigaciones de carácter interdisciplinario.

\noindent En este marco, la presente investigación justifica su relevancia al proponer la construcción de un sistema computacional integral orientado al procesamiento y análisis de bioseñales electromiográficas mediante Python. Este sistema permitirá aplicar la Transformada Rápida de Fourier (FFT) bajo condiciones de parametrización controlada —ventana de Hamming, unidades de visualización en Vpk/rHz y ancho de bin de 15—, posibilitando la obtención de representaciones espectrales de alta resolución y la comparación estadística de los patrones de actividad registrados en distintas regiones uretrales y fases temporales. Con ello, se pretende dotar a los investigadores de una herramienta reproducible, flexible y escalable, que reduzca la dependencia de plataformas cerradas y promueva la innovación en el tratamiento de datos biomédicos.

\noindent Desde una perspectiva epistemológica, este trabajo contribuye al fortalecimiento del vínculo entre la matemática aplicada, la computación científica y la neurofisiología experimental, consolidando un enfoque interdisciplinario indispensable para el avance del conocimiento biológico. La creación de un entorno computacional reproducible no solo amplía las capacidades analíticas del laboratorio, sino que también promueve la formación de nuevas generaciones de investigadores con competencias en programación científica, análisis de datos y modelación matemática. De esta manera, la investigación trasciende el ámbito técnico para insertarse en el contexto más amplio de la ciencia abierta, fomentando la colaboración, la transparencia y la sostenibilidad del conocimiento.

\noindent Finalmente, la pertinencia de esta investigación radica en su doble impacto: científico y tecnológico. Desde el punto de vista científico, ofrece una metodología rigurosa para el análisis de la actividad electromiográfica uretral, capaz de revelar patrones espectrales y variaciones fisiológicas con implicaciones clínicas potenciales en el estudio de disfunciones urinarias. Desde el punto de vista tecnológico, promueve el desarrollo de herramientas computacionales accesibles y adaptables a múltiples escenarios experimentales, contribuyendo a la independencia tecnológica y a la consolidación de prácticas de investigación reproducibles dentro del Centro Tlaxcala de Biología de la Conducta y de la comunidad científica en general.

\noindent En síntesis, la justificación de este trabajo reside en la necesidad de articular el conocimiento fisiológico con el rigor matemático y la eficiencia computacional, generando una herramienta innovadora que permita explorar, desde una perspectiva cuantitativa y reproducible, los complejos mecanismos de control neuromuscular de la uretra. Con ello, se pretende no solo enriquecer el análisis experimental existente, sino también sentar las bases para futuras investigaciones en el ámbito del procesamiento de bioseñales, el análisis espectral avanzado y la aplicación de técnicas computacionales al estudio de la función neuromuscular.

\section{Objetivos}

\subsection{Objetivo general}

\noindent Desarrollar un marco computacional integral implementado en el lenguaje de programación Python, orientado al procesamiento, análisis y visualización de bioseñales electromiográficas provenientes de estudios experimentales del esfínter externo de la uretra en conejas, con el propósito de establecer una metodología reproducible, transparente y adaptable que permita la caracterización espectral y estadística de dichas señales. Este marco deberá integrar algoritmos de transformación digital —particularmente la Transformada Rápida de Fourier (FFT) parametrizada mediante ventana de Hamming, unidades de visualización en Vpk/rHz y ancho de bin definido—, junto con herramientas de análisis matemático-estadístico y de representación gráfica que posibiliten la identificación de patrones diferenciales en la actividad neuromuscular bajo condiciones de estimulación controlada. Asimismo, el sistema computacional propuesto buscará consolidarse como una plataforma abierta y escalable, capaz de fortalecer la investigación interdisciplinaria entre la matemática aplicada, la computación científica y la neurofisiología experimental, contribuyendo al desarrollo de nuevas estrategias de interpretación cuantitativa en el estudio de bioseñales y a la consolidación de prácticas científicas sustentadas en los principios de eficiencia, reproducibilidad y ciencia abierta.

\subsection{Objetivos específicos}

\begin{itemize}
	\item Diseñar e implementar un algoritmo computacional en Python que permita el procesamiento estructurado de bioseñales electromiográficas del esfínter externo de la uretra, garantizando la trazabilidad y consistencia de los datos experimentales obtenidos en el Centro Tlaxcala de Biología de la Conducta (CTBC). Este algoritmo deberá incluir rutinas para la lectura, segmentación y preprocesamiento de los archivos de registro, optimizando el manejo de grandes volúmenes de información mediante técnicas de filtrado y normalización de señales.
	
	\item Aplicar la Transformada Rápida de Fourier (FFT) bajo condiciones específicas de parametrización —utilizando ventana de Hamming, unidades de visualización en Vpk/rHz y un ancho de bin de 15— con el propósito de obtener una caracterización espectral precisa de las bioseñales. Esta implementación deberá posibilitar la identificación de componentes frecuenciales representativos, reduciendo el leakage espectral y mejorando la resolución de los datos para su análisis estadístico posterior.
			
	\item Desarrollar herramientas metodológicas y visuales basadas en bibliotecas científicas de Python —tales como NumPy, SciPy, Matplotlib y Pandas— que permitan representar las bioseñales en los dominios temporal y frecuencial, favoreciendo una interpretación más profunda de la dinámica eléctrica subyacente. Estas herramientas deberán integrar gráficos interactivos, espectrogramas y representaciones comparativas entre regiones uretrales y fases de estimulación.
	
	\item Implementar módulos de análisis estadístico y matemático que faciliten la comparación de los valores espectrales obtenidos entre distintas regiones anatómicas de la uretra (proximal, medial y distal) y entre las ventanas temporales de análisis (Basal, A-inicial, A-mayor y A-final). Dichos módulos deberán permitir evaluar la existencia de diferencias significativas en los patrones de actividad neuromuscular, contribuyendo a la validación fisiológica de los resultados y al fortalecimiento de las inferencias experimentales.
	
	\item Construir un entorno computacional reproducible y abierto, disponible tanto en modalidad de línea de comandos (CLI) como en cuadernos interactivos (Jupyter Notebooks), que integre los algoritmos, rutinas y visualizaciones desarrolladas. Este entorno deberá estar debidamente documentado, con un enfoque en la transparencia metodológica, la adaptabilidad a otros tipos de bioseñales (como EEG o ECG), y la promoción de buenas prácticas científicas en el análisis computacional de datos fisiológicos.
	
	\item Validar el desempeño y la fiabilidad del marco computacional desarrollado mediante la comparación de los resultados obtenidos con aquellos reportados en investigaciones previas (véase [Corona-Quintanilla et al., año]) y con análisis realizados mediante software especializado de referencia. Esta etapa permitirá verificar la precisión, eficiencia y reproducibilidad del sistema, garantizando su utilidad científica en la caracterización cuantitativa de la actividad electromiográfica uretral.
	
\end{itemize}

\section{Hipótesis y preguntas de investigación}

\subsection{Hipótesis}

\noindent Se plantea que el diseño e implementación de un marco computacional en Python, basado en la Transformada Rápida de Fourier (FFT) y en técnicas avanzadas de procesamiento digital de señales, permitirá caracterizar con precisión y reproducibilidad la actividad electromiográfica del esfínter externo de la uretra (EEU) en conejas sometidas a estimulación eléctrica controlada.

\noindent De manera específica, se postula que la aplicación de la FFT parametrizada mediante ventana de Hamming, unidades de visualización en Vpk/rHz y ancho de bin de 15, posibilitará la identificación cuantitativa de componentes espectrales diferenciables entre las regiones proximal, medial y distal de la uretra, así como entre las ventanas temporales Basal, A-inicial, A-mayor y A-final, reflejando variaciones fisiológicas coherentes con los patrones de activación neuromuscular reportados en modelos experimentales previos (véase [Corona-Quintanilla et al., año]).

\noindent En consecuencia, si el sistema computacional desarrollado logra reproducir y visualizar tales diferencias con fidelidad y consistencia estadística, entonces será posible afirmar que el empleo de un entorno computacional abierto y matemáticamente fundamentado no solo optimiza el análisis de bioseñales, sino que además constituye una alternativa metodológica superior frente al uso de software propietario, al garantizar mayor transparencia, flexibilidad y rigor científico en el estudio de los procesos neurofisiológicos que regulan la función uretral.

\subsection{Preguntas de investigación}

\begin{enumerate}
	\item ¿De qué manera puede un marco computacional desarrollado en Python optimizar el procesamiento y análisis de bioseñales electromiográficas provenientes del esfínter externo de la uretra en modelos experimentales animales?
	
	El desarrollo de un marco computacional en Python, sustentado en algoritmos abiertos y reproducibles, permite optimizar el procesamiento de bioseñales mediante la integración de rutinas automatizadas de lectura, segmentación, transformación espectral y análisis estadístico, reduciendo la dependencia de software propietario y garantizando la trazabilidad y transparencia del proceso analítico.
	
	\item ¿Qué ventajas ofrece la aplicación de la Transformada Rápida de Fourier (FFT) en la caracterización espectral de bioseñales electromiográficas respecto a los métodos tradicionales de análisis temporal?
	
	La FFT posibilita descomponer la señal en sus componentes frecuenciales, revelando patrones oscilatorios y estructuras espectrales no perceptibles en el dominio temporal. Ello proporciona una descripción más completa y cuantitativa de la actividad neuromuscular, incrementando la sensibilidad para detectar variaciones fisiológicas entre regiones anatómicas y momentos experimentales.
	
	\item ¿Cómo contribuye la parametrización controlada de la FFT —mediante ventana de Hamming, unidades en Vpk/rHz y ancho de bin definido— a la precisión y reproducibilidad del análisis espectral?
	
	La correcta parametrización de la FFT reduce el leakage espectral, mejora la resolución en frecuencia y permite la comparación objetiva de resultados entre distintas sesiones experimentales. De este modo, se asegura una representación espectral coherente y reproducible, indispensable para establecer inferencias fisiológicas confiables.
	
	\item ¿En qué medida la integración de herramientas matemáticas, estadísticas y de visualización en Python favorece la interpretación y validación de resultados en el análisis de bioseñales?
	
	La combinación de bibliotecas científicas como NumPy, SciPy, Matplotlib y Pandas permite construir un entorno analítico integral, capaz de procesar, representar y comparar señales en los dominios temporal y frecuencial. Esto facilita la validación cruzada de resultados y la identificación de correlaciones significativas entre parámetros espectrales y respuestas fisiológicas.
	
	\item ¿De qué forma la aplicación de un marco computacional abierto y reproducible puede fortalecer la investigación neurofisiológica en el ámbito de la continencia urinaria?
	
	El uso de entornos de código abierto promueve la colaboración interdisciplinaria, la transparencia metodológica y la replicabilidad de los resultados. Además, al permitir la personalización de algoritmos y parámetros, el marco computacional contribuye al desarrollo de nuevas estrategias analíticas aplicables a otros modelos experimentales de función neuromuscular y control urinario (véase [Corona-Quintanilla et al., año]).
\end{enumerate}

\goodbreak

